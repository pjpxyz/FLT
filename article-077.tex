%%-----------------------------------------------------%%
%% This file may act as the main file of your article. %%
%% Please uncomment one of the following lines.        %%
%%-----------------------------------------------------%%

%% \documentclass[CM,GP]{degruyter-crelle}       %% Equations numbered as (1), (2) etc.
\documentclass[SecEq,CM,GP]{degruyter-crelle} %% Equations numbered as (1.1), (1.2) etc.

%% Do not alter the following line.
\firstpage{}\volume{}\volumeyear{}\copyrightyear{}\doiyear{}\doi{}\received{XXX}\revised{XXX}
% \firstpage{}\volume{}\volumeyear{}\copyrightyear{}\doiyear{}\doi{}\received{XXX}\revised{XXX}



% \title[Shortened title for the running title]{Title}
\title[FLT derived in a direct computation]{Fermat's Last Theorem derived in a direct computation}

\author{Pawel J.}{Piskorz}{}{Krakow}
% \author{Firstname}{Lastname}{}{City}

\contact{DPS, Krakowska 55, 31-066, Krakow, Poland}{paweljanpiskorz@gmail.com}
% \contact{Department, university, street, zip code, city, country}{e-mail}

% \researchsupported{Please insert information concerning research grant support here.}

\researchsupported{The author received funds from PFRON (State Fund of Rehabilitation of Handicapped Persons) allowing the purchase of a laptop computer
on which this publication has been written.}


%% Define theorem-like environments as usual: 

\theoremstyle{plain}
 \newtheorem{theorem}{Theorem}[section]

\theoremstyle{definition} 
 \newtheorem{remark}[theorem]{Remark}


\begin{document}

\begin{abstract}
    We propose a procedure which allows to compute the only acceptable natural
    exponents of the positive integers $X, Y, Z$
    in the equation of the Fermat's Last Theorem. 
    We use the approach similar to the one applied in
    computing of the expected value and the standard deviation of number of successes in Bernoulli 
    trials presented by Kenneth S.~Miller.
\end{abstract}

%% \tableofcontents %% Just for papers exceeding 50 pages.



% \section{Inspiration}\label{sec:inspiration}


\section{Introduction}\label{sec:intro}

We write the equation from the Fermat's Last Theorem~\cite{FLT_Wikipedia}
\begin{equation}
\label{eq:Fermat}
X^N + Y^N = Z^N
\end{equation}
in which
$X, Y, Z \in \mathbb{Z_{+}}$ are integers
greater than zero
and $N \in \mathbb{N}$ is a natural number.

Without loss of generality we rewrite Equation~(\ref{eq:Fermat}) as
\begin{equation}
  \label{eq:FermatGeneralEquation}
  (x + Mpx)^{N} qr + (y + Lqy)^{N} rp = (z + Krz)^{N} pq
  \end{equation}
with $p, q, r \in \mathbb{R}$
and $M, L, K \in \mathbb{N}$\@.
Now we set $q=1$ and $r=1$
with $Y = y + Ly$ and $Z = z + Kz$
obtaining
\begin{equation}
\label{eq:FermatEquation}
(x + Mpx)^N + Y^N p = Z^N p
\end{equation}
with $X=x + Mpx$.
We have introduced a real number parameter $p \in \mathbb{R}$ the value of which will 
be later set to 1\@.
The number $M \in \mathbb{N}$ is a natural number.
With such assumptions we must have $x \in \{ \frac{1}{M+1}, \frac{2}{M+1}, \frac{3}{M+1}, \ldots \}$
in order for the variable $X$ to assume integer values.

E.g.:

if $M=1$, $p=1$, $x \in \{ \frac{1}{2}, \frac{2}{2}, \frac{3}{2}, \ldots \}$
 then $X = \{\frac{1}{2} + 1 \cdot \frac{1}{2}, \frac{2}{2} + 1 \cdot \frac{2}{2}, \frac{3}{2} + 1 \cdot \frac{3}{2}, \ldots \}$
what gives 
$X \in \{1, 2, 3, \ldots \}$;

if $M=7$, $p=1$, $x \in \{ \frac{1}{8}, \frac{2}{8}, \frac{3}{8}, \ldots \}$
 then $X = \{\frac{1}{8} + 7 \cdot \frac{1}{8}, \frac{2}{8} + 7 \cdot \frac{2}{8}, \frac{3}{8} + 7 \cdot \frac{3}{8}, \ldots \}$
what gives again
$X \in \{1, 2, 3, \ldots \}$\@.

We always arrive at $X \in \{1, 2, 3, \ldots \}$ for any natural number $M$ and $p=1$,
i.e.~we have ensured that always $X \in \{1, 2, 3, \ldots \}$ no matter which natural number $M$ we take
into account. Similarly we can ensure that $Y$ and $Z$ are also integers greater than zero
appropriately choosing natural number parameters $L$ and $K$ and appropriately real numbers $q$ and $r$\@.


\section{Computations}\label{sec:computations}

We take partial derivative of both sides of Equation~(\ref{eq:FermatEquation}) with respect to $p$ 
\begin{equation}
\label{eq:FermatEquationTakingDerivative}
\frac{\partial (x + Mpx)^N}{\partial p} + \frac{\partial Y^N p}{\partial p} = \frac{\partial Z^N p}{\partial p}
\end{equation}
We compute the partial derivtive of $(x + Mpx)^N$ as follows
\begin{eqnarray}
\frac{\partial (x + Mpx)^N}{\partial p}
=
N(x + Mpx)^{N-1}  \frac{\partial (x + Mpx)}{\partial p}   \\ \nonumber
=
N(x + Mpx)^{N-1} Mx 
=
NM  (Mp+1)^{N-1} x^{N-1}  x     \\  \nonumber
=
NM  (Mp+1)^{N-1}  x^N
\end{eqnarray}
% If we set p=1 then we receive
We receive a new equation
\begin{equation}
NM  (Mp+1)^{N-1}  x^N + Y^{N} = Z^{N}
\end{equation}
in which we can set the prameter $p=1$ obtaining
\begin{equation}
\label{eq:AfterPartialDerivativeWith_p_as_one}
NM  (M+1)^{N-1}  x^N + Y^{N} = Z^{N}
\end{equation}
If we set $p=1$ in Equation~(\ref{eq:FermatEquation}) on the other hand we obtain
\begin{equation}
\label{eq:two_to_compare}
(M+1)^{N} x^{N} + Y^N  = Z^N
\end{equation}
We compare the coefficients at the term with $x^{N}$ in Equations~(\ref{eq:AfterPartialDerivativeWith_p_as_one}) 
and~(\ref{eq:two_to_compare})
receiving a constraint equation for $N$
\begin{equation}
\label{eq:constraint_for_N}
NM(M+1)^{N}/(M+1) = (M+1)^{N}
\end{equation}
and therefrom we obtain the values of exponent $N$ as a function of $M$
\begin{equation}
  \label{eq:constraintNM}
  N(M) = \frac{M+1}{M}
\end{equation}
Quite similarly we can arrive at the formulas for $N$ as a function of $L$
\begin{equation}
  \label{eq:constraintNL}
  N(L) = \frac{L+1}{L}
\end{equation}
and for $N$ as a function of $K$
\begin{equation}
   \label{eq:constraintNK}
  N(K) = \frac{K+1}{K}
\end{equation}
% what gives the constraint equations for the functions $N(L)$ and $N(K)$ similar to the constraints in Equations~\ref{eq:constraint}
% for the function $N(M)$\@.



\section{Conclusion}\label{sec:conclusion}

We have started with the Equation~(\ref{eq:Fermat}) with assumption that $X, Y, Z$ are positive 
integers and $N$ is a natural number.
We received the system of three constraint Equations~(\ref{eq:constraintNM}), (\ref{eq:constraintNL}) and (\ref{eq:constraintNK})
for the four unknowns $M, L, K$ and the number $N$\@.
It means that one unknown among the four ones must assume two integer values. We see below that it is the variable $N$\@.

For $N(M)$ we have
\begin{eqnarray}
\label{eq:constraint}
N(M=1) = \frac{2}{1} = 2    \\ \nonumber
N(M=2) = \frac{3}{2}        \\ \nonumber
N(M=3) = \frac{4}{3}        \\ \nonumber
N(M=4) = \frac{5}{4}        \\ \nonumber
\vdots                      \\ \nonumber
N(M=\infty) = 1             \\ \nonumber
\end{eqnarray}
% We need to accept only the natural number solutions $N(M)=1$ and $N(M)=2$ according to our assumptions.
% Similar computations lead to the expressions for $N(L)$ and $N(K)$\@. 

% In those results we have to reject the solutions which are not natural numbers.

Quite similarly we can compute $N(L)$ and N(K)\@.

We can state that with our assumption of having $N \in \mathbb{N}$ we have to reject
all $N(M)$, $N(L)$ and $N(K)$ solutions which are not natural numbers.
Our fourth unknown is $N$ equal to either $1$ or $2$ what is in perfect agreement with the Fermat's Last Theorem.


% \acknowl{Insert acknowledgments of the assistance of colleagues or similar notes of appreciation here.}


\acknowl{
    This article is in honor of American mathematician Kenneth~S.\ Mil\-ler. Due to his 
    technique of computing the expected value and the standard deviation of number of 
    successes in Bernoulli trials~\cite{book_by_Miller}
    we were able to obtain our results. 
    Author would also like to thank an
    anonymous student without whom it would take longer to
    write this paper and who suggested placing the number $p$ next to symbols $Y^{N}$ and $Z^{N}$
    in Equation~(\ref{eq:FermatEquation}) while the author was working on the universal expression 
    of the natural number $X$ using rational number $x$ and a real parameter $p$ for partial differentiation.    
}



% \section{Appendix}\label{sec:appendix}



%-------------------------------------------------------------------------%
% GENERAL INSTRUCTIONS                                                    %
% Please keep formatting macros to a minimum.                             %
% Avoid redundant source code such as unused definitions or longer        %
% passages of comments (% or {comment}).                                  %
% Do not worry about bad page breaks, etc. and avoid adding extra space   %
% and using glue to improve the appearance of the manuscript.             %
% Please ensure that the material has been carefully proofread and that   %
% all files used are made available to the editors/publisher.             %
%-------------------------------------------------------------------------%

%-------------------------------------------------------------------------%
% PACKAGES                                                                %
% Missing packages can be downloaded at www.tug.org/ctan.html.            %
%-------------------------------------------------------------------------%

%-------------------------------------------------------------------------%
% MULTILINE EQUATIONS                                                     %
% For numbered/unnumbered multiline equations use align/align* instead of %
% eqnarray. (Other available environments for multiline displays are      %
% gather, multline, aligned, split, etc.)                                 %
%-------------------------------------------------------------------------%

%-------------------------------------------------------------------------%
% QED SYMBOL                                                              %
% The \qedhere command forces the QED symbol to go on the present line,   %
% which might be useful if the last part of a proof environment is, e.g., %
% a displayed equation or list environment.                               %
%-------------------------------------------------------------------------%

%-------------------------------------------------------------------------%
% SOME MORE MATH                                                          %
% -- Avoid blank lines before or after a display, unless you really want  %
%    to start a new paragraph.                                            %
% -- Use equation*, gather* or the bracket pair \[ and \] instead of      %
%    double dollar signs $$.                                              %
% -- Use \operatorname{...} for math operators that are not predefined    %
%    or define new operators with \DeclareMathOperator. lim-like          %
%    operators can be defined with \DeclareOperator*                      %
% -- Use \quad for spacings in displayed formulas.                        %
% -- Use \substack{... \\ ...} for multiline subscripts on, e.g., sums.   %
%-------------------------------------------------------------------------%

%-------------------------------------------------------------------------%
% ENUMERATIONS                                                            %
% By default, the labels of first level enumerations are (i), (ii), etc.  %
% You may change them to become, e.g., A., B. etc., using an optional     %
% argument: \begin{enumerate}[A.]                                         %
%-------------------------------------------------------------------------%

%-------------------------------------------------------------------------%
% FIGURES AND TABLES                                                      %
% You may include a figure writing                                        %
% \begin{figure}[t]                                                       % 
%   \includegraphics[width=.8\textwidth]{xxx.yyy}                         %
%   \caption{Caption.}\label{fig:zzz}                                     %
% \end{figure}                                                            %
% A table is included similar using the table environment.                %
% Sub-captions are created as follows:                                    %
% \begin{figure}[t]                                                       %
%   \subfloat[Caption a]{\includegraphics[width=.45\textwidth]{...}}      %
%   \hfill                                                                %
%   \subfloat[Caption b]{\includegraphics[width=.45\textwidth]{...}}      %
%   \caption{Caption.}                                                    %
% \end{figure}                                                            %
%-------------------------------------------------------------------------%

%-------------------------------------------------------------------------%
% QUOTATION MARKS                                                         %
% Double quotation marks are produced using `` and ''. The opening quote  %
% then looks like a superscript 66 and the closing quote like a           %
% superscript 99.                                                         %
%-------------------------------------------------------------------------%

%-------------------------------------------------------------------------%
% HYPHEN VS. DASH                                                         %
% The hyphen (-) is used for compound words like $n$-dimensional.         %
% (Don't write things like $n-$dimensional!)                              %
% The en-dash (--) is used for number ranges and it can stand for `and',  %
% e.g., between two names: ``Various generalizations of the               %
% Cauchy--Schwarz inequality are presented on pages 63--67.''             %
% To partition a sentence one may use the em-dash (---) with no space     %
% on either side; we prefer the en-dash with a blank on both sides.       %
% The mathematical minus sign is produced by a single - within            %
% mathematics mode.                                                       %
%-------------------------------------------------------------------------%

%-------------------------------------------------------------------------%
% REFERENCES                                                              %
% Use \label and \ref for all cross-references to equations, figures,     %
% tables, sections, subsections, etc.                                     %
%-------------------------------------------------------------------------%

%-------------------------------------------------------------------------%
% BIBLIOGRAPHY                                                            %
% References should be collected at the end of the paper and numbered in  %
% alphabetical order of the authors' names. Titles of journals should be  %
% abbreviated as in Mathematical Reviews. Please do not use small caps    %
% for the names of the authors. The preferred style is shown in the       %
% examples below.                                                         %
% Refer to the bibliography entries using \cite commands.                 %
% Using BibTeX, you simply have to add the names of your databases as     %
%   \bibliography{mydatabase}                                             %
% The file crelle.bst then ensures the correct style.                     %
%-------------------------------------------------------------------------%

\begin{thebibliography}{9}

% \bibitem{article}
% \textit{T.~Angel} and \textit{I.~E.~Shparlinski},
% Article in a journal, J. reine angew. Math. \textbf{647} (2010), 123--156.

% \bibitem{preprint}
% \textit{C.~Bonnaf\'e},
% Preprint, preprint 2008, \url{http://arxiv.org/abs/0805.4100}.

\bibitem{FLT_Wikipedia}
\textit{Fermat's Last Theorem},
\url{https://en.wikipedia.org/wiki/Fermat%27s_Last_Theorem}.



% https://en.wikipedia.org/wiki/Fermat%27s_Last_Theorem




% \bibitem{proceedings}
% \textit{P. Brooksbank},
% Article in a conference proceedings,
% in: Finite geometries (Pingree Park 2004), Oxford University Press, Oxford (2006), 1--16.

% \bibitem{phdthesis}
% \textit{J.~D.~King},
% Unpublished dissertation,
% Ph.D. thesis, University of Cambridge, 1995.

% \bibitem{book}
% \textit{F. Sukochev}, \textit{S. Lord} and \textit{D. Zanin},
% Book, 2nd ed., De Gruyter, Berlin 2012.

\bibitem{book_by_Miller}
\textit{Kenneth S.~Miller},
Engineering Mathematics, Dover Publications, Inc.\ 1956.




\end{thebibliography}

\end{document}
